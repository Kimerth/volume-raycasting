Vizualizarea imaginilor medicale volumetrice este importantă atât pentru diagnosticarea pacienților de către cadre medicale specializate, cât și pentru studenții la facultatea de medicină. Deoarece volumul de date este mare și structurat în trei dimensiuni, acesta poate fi vizualizat doar în două dimensiuni, astfel vizualizarea acestui tip de date este dificilă. În această lucrare am prezentat implementarea unor tehnici pentru redarea imagnilor semnificative în care regiunile de interes pot fi vizibile și bine delimitate.

Folosind volume din setul de date CT-ORG în aplicația de vizualizare, se pot observa detalii în scanările respective, diferite organe pot fi identificate, și având în vedere că unele scanări provin de la pacienți cu tumori maligne, în unele dintre aceste cazuri pot fi observate astfel de probleme. Este dificil de creat o funcție de transfer care redă imaginea într-un mod realist sau de înaltă calitate, dar poate fi creată o funcție de transfer care îmbunătățește substanțial vizualizarea. Rezultatele segmentării semantice automate, în cele mai multe cazuri, nu conțin erori foarte evidente, iar dacă acestea există, pot fi eliminate prin aplicarea algoritmului de netezire. În urma antrenării, aplicând modelul de segmentare pe datele de test, au rezultat măști de segmentare asemănătoare cu cele create manual de către autorii setului de date.

Dificultățile întâmpinate în implementarea soluției dorite sunt, printre altele: redarea imaginilor volumetrice în mod eficient cu suficiente detalii, construirea unei interfețe ce permite crearea funcțiilor de transfer unidimensionale, antrenarea unei rețele neuronale pentru segmentarea semantică a datelor volumetrice și încărcarea acesteia în aplicația de vizualizare, aplicarea măștii de segmentare și salvare în memoria nevolatilă a funcției de transfer și a măștii de segmentare și citirea și încărcarea în memorie a datelor volumetrice.

Fiind o sarcină dificilă, metodele implementate pot fi îmbunătățite, sau pot fi implementate metode diferite care să îmbunătățească vizualizarea în cadrul imagisticii medicale. Implementarea funcțiilor de transfer multidimensionale este o primă direcție de dezvoltare, și constă în cercetarea, implementarea și testarea modalităților de creare a funcțiilor de transfer multidimensionale. De asemenea, poate fi luată în considerare și implementarea unei metode de a modifica manual rezultatul segmentării automate, deoarece metodele folosite pentru calcularea măștii de segmentare nu sunt perfecte.
