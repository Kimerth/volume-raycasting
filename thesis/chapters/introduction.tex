Datele volumetrice sunt un set de eșantioane ce reprezintă una sau mai multe proprietăți ale unui punct într-o locație tridimensională. Vizualizarea volumelor constă în redarea unei proiecții 2D a unui set de date 3D. Una dintre metodele populare pentru această sarcină este ray casting. Aceasta este un tip de redare directă a volumelor (DVR, Direct Volume Rendering) ce constă în emiterea de raze dinspre punctul de observație prin volum și eșantionarea valorilor în puncte echidistante, rezultatul final fiind compunerea acestor valori. Funcțiile de transfer sunt funcții ce transformă valorile scalare stocate în seturile de date volumetrice pentru a îmbunătăți vizualizarea acestora.

Sarcina redării imaginilor volumetrice este mai dificilă decât cea a redării unei imagini bidimensionale deoarece datele reprezentate sunt mult mai numeroase. Spre exemplu, o imagine FHD (Full High Definition) conține două milioane de valori RGB, adică șase miloane de valori scalare, iar într-un set de date volumetric medical uzual sunt două sute de milioane de valori scalare. Unitățile de procesare grafică (GPU) au avansat în ultimele decenii, iar în momentul actual această sarcină poate fi realizată folosind chiar și procesoare grafice pentru consumatori.

Implementarea propusă în această lucrare se bazează pe OpenGL pentru redarea datelor volumetrice și ImGui pentru interfața grafică. Algoritmul de ray casting este implementat într-o singură etapă, calculând intersecția razelor cu un AABB\footnote{Casetă de delimitare aliniată pe axă sau axis aligned bounding box} și compunând culorile și transparențele din funcția de transfer pentru valorile intensităților în puncte echidistante folosind tehnica back-to-front. Punctele extreme ale AABB-ului sunt obținute din interfață, astfel putând fi excluse segmente din volumul de date, obtinandu-se un efect similar cu clipping planes aliniat la axele de referință. Pentru a vizualiza regiunile de interes utilizatorul poate aplica transformări de rotație, scalare și translație. Pentru acestea pot fi folosite atât mouse-ul cât și tastatura.

Pentru obținerea unei măști de segmentare în mod automat poate fi utilizată o rețea neuronală antrenată pentru această sarcină, iar pentru determinarea arhitecturii și parametrilor potriviți vor fi necesare multiple experimente. În acest sens a fost implementată o aplicație care să faciliteze antrenarea rețelelor atât local cât și folosind servicii de tip PaaS\footnote{Platform as a service reprezintă un tip de serviciu cloud ce permite folsoirea unui mediu de dezvoltare online.} folosind fișiere de configurare ce pot fi suprascrise din lina de comandă, astfel putând fi create modele diferite cu modificări minimale ale codului sursă. Odată obținut un model viabil, acesta trebuie încărcat în aplicația de vizualizare și, utilizând preprocesări și postprocesări asemănătoare antrenării, poate fi redată masca de segmentare.


În capitolul \ref{ch:fundamentals} sunt prezentate comparativ realizări actuale pe aceeași temă și sunt descrise specificațiile privind caracteristicile așteptate de la aplicație. De asemenea sunt prezentate aspecte teoretice necesare pentru proiectarea și implementarea aplicației.

În capitolul \ref{ch:design} este descrisă proiectarea aplicației de redare, sunt prezentați algoritmii principali și sunt descrise componentele pe care le-am dezvoltat și implementat. Mai este descris și framework-ul cu care vor fi efectuate experimente pentru antrenarea rețelei neuronale și modalitatea în care aceasta va fi utilizată în aplicația de redare.
Capitolul \ref{ch:implementation} descrie funcționalitatea sistemului și cum au fost implementate componentele principale. Este descris și modul în care sunt încărcate și salvate funcțiile de transfer și măștile de segmentare.

Capitolul \ref{ch:results} începe prin prezentarea modalității de compilare a aplicației și a punerii în funcționare a acesteia. Apoi sunt prezentate interacțiunile pe care le poate avea utilizatorul cu aplicația și sunt discutate câteva exemple de vizualizare a unor imagini volumetrice din diferite unghiuri și diferite niveluri de zoom. Spre finalul acestui capitol sunt prezentate rezultate experimentale pentru antrenarea rețelei neuronale folosită pentru segmentarea semantică, și sunt analizate și rezultatele aplicării acestui model în aplicația de redare.
Lucrarea se încheie cu concluziile referitoare la rezultatele obținute și la utilitatea aplicației propuse, împreună cu eventuale direcții de dezvoltare.
